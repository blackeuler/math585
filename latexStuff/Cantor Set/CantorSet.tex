\documentclass[12pt]{article}
\usepackage{amssymb,amsmath,amsthm,fullpage}
\usepackage{graphicx}

\theoremstyle{plain}
\newtheorem{theorem}{Theorem}
\newtheorem{lemma}{Lemma}
\newtheorem{proposition}{Proposition}
\newtheorem{problem}{Problem}

%The above commands create numbered environments for theorems, lemmas, propositions, and problems.
%Use as needed.


\newtheorem*{remark}{Remark}
\newtheorem*{solution}{Solution}
\newtheorem*{example}{Example}

%These commands (with asterisk) create un-numbered theorem-like environments.   Use as needed.


\title{Cantor Set}
\author{Christopher D. Miller}
\date{February 5 2019}

% These lines produce reasonably formatted title information
% for your assignment once \maketitle is invoked in the body of the document.
% You can use the command \date{\today} to automatically generate the date the file is compiled.

\usepackage{hyperref}

\begin{document}

\maketitle

Say we have the closed interval $[0,1].$ Lets define a set $C_1$ defined by taking the middle third out of the closed interval $[0,1]$. So we have the $C_1$ consists of the closed intervals $\big[0,\frac{1}{3}\big]$ and $\big[\frac{2}{3},1\big]$. Now lets repeat this process on $C_1$ to obtain the set $C_2$. Taking the middle third out of the closed intervals $\big[0,\frac{1}{3}\big]$ and $\big[\frac{2}{3},1\big]$ leaving us with the closed intervals $\big[0,\frac{1}{9}\big]$, $\big[\frac{2}{9},\frac{1}{3}\big]$,$\big[\frac{2}{3},\frac{7}{9}\big]$, $\big[\frac{8}{9},1\big]$. Now for each of these closed intervals you want to repeat this process. Each iteration taking the middle third out of the closed intervals you obtain. The cantor set is the intersection of all these sets. Now an interesting fact is that this set is non-empty because the numbers 0 and 1 are included in every set. Since the set is closed.



\end{document}
