\documentclass[12pt]{article}
\usepackage{amssymb,amsmath,amsthm,fullpage}
\usepackage{graphicx}

\theoremstyle{plain}
\newtheorem{theorem}{Theorem}
\newtheorem{lemma}{Lemma}
\newtheorem{proposition}{Proposition}
\newtheorem{problem}{Problem}

%The above commands create numbered environments for theorems, lemmas, propositions, and problems.
%Use as needed.


\newtheorem*{remark}{Remark}
\newtheorem*{solution}{Solution}
\newtheorem*{example}{Example}

%These commands (with asterisk) create un-numbered theorem-like environments.   Use as needed.


\title{Riemann Hypothesis}
\author{Christopher D Miller}
\date{March 27 2019}

% These lines produce reasonably formatted title information
% for your assignment once \maketitle is invoked in the body of the document.
% You can use the command \date{\today} to automatically generate the date the file is compiled.

\usepackage{hyperref}

\begin{document}

\maketitle

In 1859, a german mathematician, Bernard Riemann, postulates that the Riemann Zeta Function only has zeroes at its even negative numbers. This comes after he ontains a formula for the number of primes up to a certain limit. This formula utilizes the zeroes of the Riemann Zeta function. A simple version of the Riemann Zeta function can be expressed by the following:
$$ \zeta(s) = \sum_{n=1}^{\infty} \frac{1}{n^s}. $$
The domain for this function is all real numbers greater than 1. Although this can be expanded by methods of analytic continuation so that the function has a domain for negative integers. There are many interesting facts surrrounding the Riemann Hypothesis. Other than being an unsolved mathematical conjecture it is also challenge to win 1 million dollars. Offered by the Clay Mathematics Institute in 2000, the Riemann Hypothesis was dubbed a millenium problem. Another interesting fact about the Hypothesis is if solved it could give us more information on the distribution of prime numbers. Some believe with a proof of the Riemann Hypothesis then there will be some security flaws with RSA Encryption methods. 



\begin{thebibliography}{9}
    \bibitem{clay}
    Bombieri, Enrico (2000), The Riemann Hypothesis – official problem description (PDF), Clay Mathematics Institute, retrieved 2008-10-25 Reprinted in (Borwein et al. 2008).
\end{thebibliography}

\end{document}
