\documentclass[12pt]{article}
\usepackage{amssymb,amsmath,amsthm,fullpage}
\usepackage{graphicx}

\theoremstyle{plain}
\newtheorem{theorem}{Theorem}
\newtheorem{lemma}{Lemma}
\newtheorem{proposition}{Proposition}
\newtheorem{problem}{Problem}

%The above commands create numbered environments for theorems, lemmas, propositions, and problems.
%Use as needed.


\newtheorem*{remark}{Remark}
\newtheorem*{solution}{Solution}
\newtheorem*{example}{Example}

%These commands (with asterisk) create un-numbered theorem-like environments.   Use as needed.


\title{Tessellations}
\author{Christopher D Miller}
\date{February 5 2019}

% These lines produce reasonably formatted title information
% for your assignment once \maketitle is invoked in the body of the document.
% You can use the command \date{\today} to automatically generate the date the file is compiled.

\usepackage{hyperref}

\begin{document}

\maketitle



\begin{theorem}
    There are exactly three regular tessellations of the plane.
\end{theorem}
\begin{proof}
    A regular tessallation is a being able to cover a plane with regular polygons without any gaps or overlaps. A regular polygon is a polygon with the same length and every angle is equal. To cover a plane without gaps then the shapes interior angle has to divide 360.The formula for the interior angle of a regular polygon is $$\frac{180(n-2)}{n} \text{where } n \text{ is the number of sides}.$$
    So we need the interior angle of the regular polygon to divide 360. In other words there exist a $k$ such that $$\frac{180(n-2)}{n}k = 360 .$$. 
    \begin{align*}
        \frac{180(n-2)}{n}k& = 360\\
        \frac{(n-2)}{n}k& = 2\\
        (n-2)k& = 2n\\
        kn-2k& = 2n\\
        kn-2k + 4 &= 2n+4\\
        (n-2)(k-2) &=4
    \end{align*}
    In order to solve this equation notice that $(n-2)\leq 4$ and $(k-2)\leq 4$. This gives us $3\leq n \leq 6$ and $3\leq n \leq 6$. The only solutions are $n = 3 ,k =6$ , $n=6,k=3$, $n=4,k=4$. So there must be only 3 regular polygons that can tesselate a plane. 
\end{proof}





\end{document}
