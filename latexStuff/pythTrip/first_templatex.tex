\documentclass[12pt]{article}
\usepackage{amssymb,amsmath,amsthm,fullpage}
\usepackage{graphicx}

\theoremstyle{plain}
\newtheorem{theorem}{Theorem}
\newtheorem{lemma}{Lemma}
\newtheorem{proposition}{Proposition}
\newtheorem{problem}{Problem}

%The above commands create numbered environments for theorems, lemmas, propositions, and problems.
%Use as needed.


\newtheorem*{remark}{Remark}
\newtheorem*{solution}{Solution}
\newtheorem*{example}{Example}
%These commands (with asterisk) create un-numbered theorem-like environments.   Use as needed.


\title{Pythagorean Triples}
\author{Christopher D Miller}
\date{February 5 2019}

% These lines produce reasonably formatted title information
% for your assignment once \maketitle is invoked in the body of the document.
% You can use the command \date{\today} to automatically generate the date the file is compiled.

\usepackage{hyperref}

\begin{document}

\maketitle
\begin{proposition}
    \begin{align*}
        a_{k+1} &= a_{k} +4\\ 
        b_{k+1} &= \frac{1}{2} a_{k} + b_{k} +1\\
        c_{k+1} &= \frac{1}{2}a_k + c_k +1
    \end{align*}
    Generates Pythagorean Triples of Height 8 with $a_0=20,b_0 = 21,c_0 = 29$
    
\end{proposition}
\begin{proof}     
We will prove by induction that Proposition 1 holds for all $k \geq 0$.\\\textbf{Base Case:} Our base case is when k = 0.So when k =0 , by definition our formula gives us a Pythagorean Triple of height 8.So our proposition is true in this case.\\
\textbf{Induction Step:} Let k $\geq$ 0 be given and suppose our proposition is true for $n=k$. Then  
\end{proof}


\end{document}
